\documentclass[english,twoside,openright]{report}

\usepackage[utf8]{inputenc}
\usepackage[T1]{fontenc} 
\usepackage{babel}

\usepackage{xcolor}% provides \colorlet
\usepackage{fixme}
\fxsetup{
    status=draft,
    author=,
    layout=inline,
    theme=color
}

\definecolor{fxnote}{rgb}{0.8000,0.0000,0.0000}
% define the background colour:
\colorlet{fxnotebg}{yellow}
\definecolor{fxfatalbg}{rgb}{0,0,100}

% refedine the layout macro:
\makeatletter
\renewcommand*\FXLayoutInline[3]{%
  \@fxdocolon {#3}{%
    \@fxuseface {inline}%
    \colorbox{fx#1bg}{\color {fx#1}\ignorespaces #3\@fxcolon #2}}}
\makeatother

\usepackage{rotating}

\usepackage{graphicx}
\usepackage{wrapfig}
\usepackage{appendix}
\usepackage{float}
\usepackage{lipsum}
\usepackage{lastpage}
\usepackage{fancyhdr}
\usepackage{varioref}
\usepackage{hyperref}    % Creates links in the PDF
\usepackage{listings}
%\usepackage{algorithmic}
\usepackage{listings}
\usepackage{tabularx}
\usepackage{subfig} %til resize af billeder
%\usepackage[usenames,dvipsnames]{color}
\definecolor{lightgray}{RGB}{232,232,232}
\usepackage{spverbatim} %Bruges til at indføre wrapping i kodemiljøer, ellers er den magen til {verbatim}
\usepackage{bussproofs} %Bruges til at indf�re derivationstr�er
\usepackage[ final ]{pdfpages}
\usepackage{epstopdf}
\usepackage[section]{placeins}

%Acronymer for P6 Giraf Projekt
\newcommand{\secref}[1]{Section \ref{#1}}
\usepackage{acronym}
\acrodef{api}[API]{Application Programming Interface}
\acrodef{cat}[CAT]{Category Administration Tool}
\acrodef{lamp}[LAMP]{Linux, Apache2, MySql and PHP}
\acrodef{giraf}[GIRAF]{Graphical Interface Resources for Autistic Folk}
\acrodef{oha}[OHA]{Open Handset Alliance}
\acrodef{gui}[GUI]{Graphical User Interface}
\acrodef{json}[JSON]{JavaScript Object Notation}
\acrodef{foss}[FOSS]{Free and Open-Source Software}

\lstdefinelanguage{cs}
  {morekeywords={abstract,event,new,struct,as,explicit,null,switch
		base,extern,object,this,bool,false,operator,throw,
		break,finally,out,true,byte,fixed,override,try,
		case,float,params,typeof,catch,for,private,uint,
		char,foreach,protected,ulong,checked,goto,public,unchecked,
		class,if,readonly,unsafe,const,implicit,ref,ushort,
		continue,in,return,using,decimal,int,sbyte,virtual,
		default,interface,sealed,volatile,delegate,internal,short,void,
		do,is,sizeof,while,double,lock,stackalloc,
		else,long,static,enum,namespace,string,MySQLTools,MySqlDataAdapter },
	  sensitive=false,
	  morecomment=[l]{//},
	  morecomment=[s]{/*}{*/},
	  morestring=[b]",
}

\lstdefinelanguage{nisse}
     {morekeywords={title,subtitle,url,@begin,@end,@setting,@u,@b,@i,@apply,@image,@title,@subtitle,@note,fade,swipe,scale,rotatescale,global,text,image,@url,@font_family,@font_color,@font_size,@font_weight},
	  sensitive=true,
}


% CODE %
\usepackage{listings}
\usepackage{color}
\usepackage{textcomp}
\definecolor{listinggray}{gray}{0.9}
\definecolor{lbcolor}{rgb}{0.9,0.9,0.9}
\lstset{
	language=nisse,
	keywordstyle=\bfseries\ttfamily\color[rgb]{0,0,1},
	identifierstyle=\ttfamily,
	commentstyle=\color[rgb]{0.133,0.545,0.133},
	stringstyle=\ttfamily\color[rgb]{0.627,0.126,0.941},
	showstringspaces=false,
	basicstyle=\small,
	numberstyle=\footnotesize,
	numbers=left,
	stepnumber=1,
	numbersep=10pt,
	tabsize=2,
	breaklines=true,
	prebreak = \raisebox{0ex}[0ex][0ex]{\ensuremath{\hookleftarrow}},
	breakatwhitespace=false,
	aboveskip={1.5\baselineskip},
  	columns=fixed,
  	upquote=true,
 	extendedchars=true,
escapeinside={(*@}{@*)},         % if you want to add a comment within your code
}


% Bibtex %
\usepackage{natbib}

% URL %
\usepackage{url}
\makeatletter
\def\url@leostyle{\@ifundefined{selectfont}{\def\UrlFont{\sf}}{\def\UrlFont{\small\ttfamily}}}
\makeatother
\urlstyle{leo}


\setlength{\headheight}{15pt}
\pagestyle{fancy}
\renewcommand{\chaptermark}[1]{\markboth{#1}{}}
\renewcommand{\sectionmark}[1]{\markright{#1}{}}
 
 
\fancyhf{} % clear header and footer
\fancyhead[LO, RE]{\textit{\rightmark}}
\fancyhead[RO, LE]{\textit{\leftmark}}


\fancypagestyle{plain}{
\fancyhead[LE,RO,RE,LO]{}
\renewcommand{\headrulewidth}{0pt}
\fancyfoot[RO,LE]{\thepage\ of \pageref{LastPage}}}
\fancyfoot[RO,LE]{\thepage\ of \pageref{LastPage}}