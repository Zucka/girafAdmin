\documentclass[a4paper,11pt,fleqn,twoside,openright,oldfontcommands]{memoir} % Brug openright hvis chapters skal starte på højresider; openany, oneside

%%%% PACKAGES %%%%
\usepackage{pageslts}
\usepackage[]{algorithm2e}
% ¤¤ Oversættelse og tegnsætning ¤¤ %
\usepackage[utf8x]{inputenc}					% Gør det muligt at bruge æ, ø og å i sine .tex-filer
\usepackage[english,danish]{babel}							% Dansk sporg, f.eks. tabel, figur og kapitel
\usepackage[T1]{fontenc}								% Hjælper med orddeling ved æ, ø og å. Sætter fontene til at være ps-fonte, i stedet for bmp					
\usepackage{latexsym}										% LaTeX symboler
\usepackage{xcolor,ragged2e,fix-cm}			% Justering af elementer
\usepackage{pdfpages}										% Gør det muligt at inkludere pdf-dokumenter med kommandoen \includepdf[pages={x-y}]{fil.pdf}	
\pretolerance=2500 											% Gør det muligt at justre afstanden med ord (højt tal, mindre orddeling og mere space mellem ord)
\usepackage{ulem}                       % Gennemstregning af ord med koden \sout{}
\usepackage{fixltx2e}										% Retter forskellige bugs i LaTeX-kernen
\usepackage[shortlabels]{enumitem}			% Muliggør enkelt konfiguration af lister
																			
% ¤¤ Figurer og tabeller – floats  ¤¤ %
\usepackage{flafter}										% Sørger for at dine floats ikke optræder i teksten før de er sat ind.
\usepackage{multirow}                		% Fletning af rækker
\usepackage{hhline}                   	% Dobbelte horisontale linier
\usepackage{multicol}         	        % Fletning af kolonner
\usepackage{colortbl} 									% Muligøre farver i tabeller
\usepackage{booktabs}										% Professionelt udseende tabeller med \toprule, \midrule og \bottomrule 
\usepackage{rotating}										% Muliggør rotation af tekst i tabeller med \begin{sideways}...\end{sideways}
\usepackage{wrapfig}										% Indsættelse af figurer omsvøbt af tekst. \begin{wrapfigure}{Placering}{Størrelse}
\usepackage{graphicx} 									% Pakke til jpeg/png billeder
\usepackage[section] {placeins}
\pdfoptionpdfminorversion=6							% Muliggør inkludering af pdf dokumenter, af version 1.6 og højere
\usepackage{rotating}	

% ¤¤ Matematiske formler og maskinkode ¤¤
\usepackage{amsmath,amssymb,stmaryrd,amsthm} 	% Bedre matematik og ekstra fonte
\usepackage{textcomp}                 	% Adgang til tekstsymboler
\usepackage{mathtools}									% Udvidelse af amsmath-pakken. 
\usepackage{eso-pic}										% Tilføj billedekommandoer på hver side
\usepackage{lipsum}											% Dummy text \lipsum[..]
\usepackage{siunitx}										% Flot og konsistent præsentation af tal og enheder med \SI{tal}{enhed}
\sisetup{decimalsymbol=comma}						% Opsætning af \SI
%\usepackage{rsphrase}										% Kemi-pakke til RS-sætninger, f.eks. \rsphrase{R1}
\usepackage[version=3]{mhchem} 					% Kemi-pakke til flot og let notation af formler, f.eks. \ce{Fe2O3}
\usepackage{listings}
\usepackage{color}

% ¤¤ Referencer, bibtex og url'er ¤¤ %
\usepackage{url}												% Til at sætte urler op med. Virker sammen med hyperref
\usepackage[danish]{varioref}						% Giver flere bedre mulighed for at lave krydshenvisninger
\usepackage{natbib}											% Litteraturliste med forfatter-år og nummerede referencer
\usepackage{xr}													% Referencer til eksternt dokument med \externaldocument{<NAVN>}

% ¤¤ Floats ¤¤ %
\let\newfloat\relax 										% Memoir har allerede defineret denne, men det gør float pakken også
\usepackage{float}

\usepackage[footnote,draft,danish,silent,nomargin]{fixme}		% Indsæt rettelser og lignende med \fixme{...} Med final i stedet for draft, udløses en error 																															for hver fixme, der ikke er slettet, når rapporten bygges.

%%%% CUSTOM SETTINGS %%%%

% ¤¤ Code Inserting ¤¤ %

\usepackage{listings}
\lstset{
language=C,                     % choose the language of the code
basicstyle=\footnotesize,       % the size of the fonts that are used for the code
numbers=left,                   % where to put the line-numbers
numberstyle=\footnotesize,      % the size of the fonts that are used for the line-numbers
stepnumber=1,                   % the step between two line-numbers. If it's 1 each line 
                                % will be numbered
numbersep=5pt,                  % how far the line-numbers are from the code
backgroundcolor=\color{white},  % choose the background color. You must add \usepackage{color}
showspaces=false,               % show spaces adding particular underscores
showstringspaces=false,         % underline spaces within strings
showtabs=false,                 % show tabs within strings adding particular underscores
frame=single,	                  % adds a frame around the code
tabsize=2,	                    % sets default tabsize to 2 spaces
captionpos=b,                   % sets the caption-position to bottom
breaklines=true,                % sets automatic line breaking
breakatwhitespace=false,        % sets if automatic breaks should only happen at whitespace
title=\lstname,                 % show the filename of files included with \lstinputlisting;
                                % also try caption instead of title
escapeinside={(*@}{@*)},         % if you want to add a comment within your code
%morekeywords={*,...}            % if you want to add more keywords to the set
inputencoding=utf8x,
extendedchars=\true
}



% ¤¤ Marginer ¤¤ %
\setlrmarginsandblock{3.5cm}{2.5cm}{*}	% \setlrmarginsandblock{Indbinding}{Kant}{Ratio}
\setulmarginsandblock{2.5cm}{3.0cm}{*}	% \setulmarginsandblock{Top}{Bund}{Ratio}
\checkandfixthelayout 									% Laver forskellige beregninger og sætter de almindelige længder op til brug ikke memoir pakker

%	¤¤ Afsnitsformatering ¤¤ %
\setlength{\parindent}{0mm}           	% Størrelse af indryk
\setlength{\parskip}{4mm}          			% Afstand mellem afsnit ved brug af double Enter
\linespread{1,1}												% Linie afstand

% ¤¤ Litteraturlisten ¤¤ %
\bibpunct[,]{[}{]}{;}{a}{,}{,} 					% Definerer de 6 parametre ved Harvard henvisning (bl.a. parantestype og seperatortegn)
\bibliographystyle{bibtex/harvard}			% Udseende af litteraturlisten. Ligner dk-apali - mvh Klein

% ¤¤ Indholdsfortegnelse ¤¤ %
\renewcommand\contentsname{Table of Contents}
\setsecnumdepth{subsection}		 					% Dybden af nummerede overkrifter (part/chapter/section/subsection)
\maxsecnumdepth{subsection}							% Ændring af dokumentklassens grænse for nummereringsdybde
\settocdepth{subsection} 								% Dybden af indholdsfortegnelsen

% ¤¤ Lister ¤¤ %
\setlist{
  topsep=0pt,														% Vertikal afstand mellem tekst og listen
  itemsep=-1ex,													% Vertikal afstand mellem items
} 

\theoremstyle{definition}
\newtheorem{definition}{Definition}
\newtheorem{example}{Example}
\newtheorem{eksempel}{Eksempel}

% ¤¤ Visuelle referencer ¤¤ %
\usepackage[colorlinks]{hyperref}			 	% Giver mulighed for at ens referencer bliver til klikbare hyperlinks. .. [colorlinks]{..}
\hypersetup{pdfborder = 0}							% Fjerner ramme omkring links i fx indholsfotegnelsen og ved kildehenvisninger ¤¤
\hypersetup{														%	Opsætning af farvede hyperlinks
    colorlinks = false,
    linkcolor = black,
    anchorcolor = black,
    citecolor = black
}

\definecolor{gray}{gray}{0.80}					% Definerer farven grå


% Opsætning af autoref%
\def\sectionautorefname{Section}
%\def\figureautorefname{Figur}
%\def\tableautorefname{Tabel}
%\def\chapterautorefname{Kapitel}
%\def\sectionautorefname{Afsnit}
%\def\subsectionautorefname{Afsnit}
\def\lstnumberautorefname{line}

% ¤¤ Opsætning af figur- og tabeltekst ¤¤ %
 	\captionnamefont{
 		\small\bfseries\itshape}						% Opsætning af tekstdelen ("Figur" eller "Tabel")
  \captiontitlefont{\small}							% Opsætning af nummerering
  \captiondelim{. }											% Seperator mellem nummerering og figurtekst
  \hangcaption													%	Venstrejusterer flere-liniers figurtekst under hinanden
  \captionwidth{\linewidth}							% Bredden af figurteksten
	\setlength{\belowcaptionskip}{10pt}		% Afstand under figurteksten
		
% ¤¤ Navngivning ¤¤ %
%\addto\captionsdanish{
%	\renewcommand\appendixname{Bilag}
%	\renewcommand\contentsname{Indholdsfortegnelse}	
%	\renewcommand\appendixpagename{Bilag}
%	\renewcommand\cftchaptername{\chaptername~}				% Skriver "Kapitel" foran kapitlerne i indholdsfortegnelsen
%	\renewcommand\cftappendixname{\appendixname~}			% Skriver "Bilag" foran bilagene i indholdsfortegnelsen
%	\renewcommand\appendixtocname{Bilag}
%}

% ¤¤ Kapiteludssende ¤¤ %
\definecolor{numbercolor}{gray}{0.7}			% Definerer en farve til brug til kapiteludseende
\newif\ifchapternonum

\makechapterstyle{jenor}{									% Definerer kapiteludseende -->
  \renewcommand\printchaptername{}
  \renewcommand\printchapternum{}
  \renewcommand\printchapternonum{\chapternonumtrue}
  \renewcommand\chaptitlefont{\fontfamily{pbk}\fontseries{db}\fontshape{n}\fontsize{25}{35}\selectfont\raggedleft}
  \renewcommand\chapnumfont{\fontfamily{pbk}\fontseries{m}\fontshape{n}\fontsize{1in}{0in}\selectfont\color{numbercolor}}
  \renewcommand\printchaptertitle[1]{%
    \noindent
    \ifchapternonum
    \begin{tabularx}{\textwidth}{X}
    {\let\\\newline\chaptitlefont ##1\par} 
    \end{tabularx}
    \par\vskip-2.5mm\hrule
    \else
    \begin{tabularx}{\textwidth}{Xl}
    {\parbox[b]{\linewidth}{\chaptitlefont ##1}} & \raisebox{-15pt}{\chapnumfont \thechapter}
    \end{tabularx}
    \par\vskip2mm\hrule
    \fi
  }
}																						% <--

\chapterstyle{jenor}												% Valg af kapiteludseende - dette kan udskiftes efter ønske

% ¤¤ Sidehoved ¤¤ %

%\makepagestyle{custom}																				% Definerer sidehoved og sidefod - kan modificeres efter ønske -->
%\makepsmarks{custom}{																						
%\def\chaptermark##1{\markboth{\itshape\thechapter. ##1}{}}		% Henter kapitlet den pågældende side hører under med kommandoen \leftmark. \itshape gør teksten kursiv
%\def\sectionmark##1{\markright{\thesection. ##1}{}}					% Henter afsnittet den pågældende side hører under med kommandoen \rightmark
%}																														% Sidetallet skrives med kommandoen \thepage	
%\makeevenhead{custom}{Gruppe B130}{}{\leftmark}							% Definerer lige siders sidehoved efter modellen \makeevenhead{Navn}{Venstre}{Center}{Højre}
%\makeoddhead{custom}{\rightmark}{}{Aalborg Universitet}			% Definerer ulige siders sidehoved efter modellen \makeoddhead{Navn}{Venstre}{Center}{Højre}
%\makeevenfoot{custom}{\thepage}{}{}													% Definerer lige siders sidefod efter modellen \makeevenfoot{Navn}{Venstre}{Center}{Højre}
%\makeoddfoot{custom}{}{}{\thepage}														% Definerer ulige siders sidefod efter modellen \makeoddfoot{Navn}{Venstre}{Center}{Højre}		
%\makeheadrule{custom}{\textwidth}{0.5pt}											% Tilføjer en streg under sidehovedets indhold
%\makefootrule{custom}{\textwidth}{0.5pt}{1mm}								% Tilføjer en streg under sidefodens indhold

%\copypagestyle{nychapter}{custom}														% Følgende linier sørger for, at sidefoden bibeholdes på kapitlets første side
%\makeoddhead{nychapter}{}{}{}
%\makeevenhead{nychapter}{}{}{}
%\makeheadrule{nychapter}{\textwidth}{0pt}
%\aliaspagestyle{chapter}{nychapter}													% <--

\pagestyle{plain}																							% Valg af sidehoved og sidefod


%%%% CUSTOM COMMANDS %%%%

% ¤¤ Billede hack ¤¤ %
\newcommand{\figur}[4]{
		\begin{figure}[H] \centering
			\includegraphics[width=#1\textwidth]{billeder/#2}
			\caption{#3}\label{#4}
		\end{figure} 
}

% ¤¤ Specielle tegn ¤¤ %
\newcommand{\grader}{^{\circ}\text{C}}
\newcommand{\gr}{^{\circ}}
\newcommand{\g}{\cdot}

% ¤¤ Promille-hack (\promille) ¤¤ %
\newcommand{\promille}{%
  \relax\ifmmode\promillezeichen
        \else\leavevmode\(\mathsurround=0pt\promillezeichen\)\fi}
\newcommand{\promillezeichen}{%
  \kern-.05em%
  \raise.5ex\hbox{\the\scriptfont0 0}%
  \kern-.15em/\kern-.15em%
  \lower.25ex\hbox{\the\scriptfont0 00}}

%%%% ORDDELING %%%%

\hyphenation{hvad hvem hvor MPa kon-se-kvens e-le-ment tem-pera-tur fjer-nes fjer-ner var-me gen-nem sys-tem kanal af-stand}

