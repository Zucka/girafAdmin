When working in a multi-project consisting of eight groups, it is important to have a common goal for the project. This chapter describes this goal as a story. Furthermore the chapter includes description of the development process and the rules of conduct.  %as well as the rules of conduct, that was used in the \ac{giraf} project of year 2013.

\section{The Goals for 2013}
Within the first couple of weeks, when all the groups had been assigned a project, a major story for the overall project was written.

\subsection*{The Major Story for 2013}
\begin{quote}
``The guardian arrives at the institution, and turns on the tablet. The guardian is aware of the arrival of a new child at the institution after lunch.
The guardian sets up and customizes a profile for the child, this includes creation of new pictograms. Furthermore the guardian prepares games and a life story for the child.

After lunch the new child and the guardian meet. The child is introduced to the communication tool Parrot. After some introduction they sit down to do some communication practice using the tool.

Afterwards the child wants to go outside to see the rest of the institution, and needs to put on some outdoor clothes. The guardian introduces the child to the Zebra tool, and together they put on the child's outdoor clothes.

When the child comes back in, the guardian and the child play the games prepared earlier by the guardian.

When they are done playing the child and the guardian read the child's life story using Tortoise.''
\end{quote}
