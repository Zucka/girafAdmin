\section{Autism}
\label{sec:autism}
Autism is a spectrum disorder, meaning that it appears in different variants and not all people who are diagnosed have the same symptoms. The disorder can often be observed within the first three years of a child's life.
Autism is a physical condition and is linked to abnormal chemistry in the brain, however the exact causes of these abnormalities are still unknown.\citep{autism}

%In general autism appears in three different variants. These are different diagnosis of autism:

%% It affects the development of some parts of the brain, especially areas concerned with social and communication skills. Autism affect how information is processed in the brain by altering how the brain's nerve cells and synapses connect. How this is occurs is not very well understood,\citep{autism} and due to this there is not existing cure for autism.

%\begin{description}
%\item [ADHD] Attention Deficit/Hyperactivity Disorder
%\item [Tourette] Verbal and motor tics
%\item [Chromosomal defects] \todo{THere is still missing something here}
%\end{description}


\subsection*{Symptoms}
\label{sub:symptoms}
Children with autism usually have difficulties understanding the concept of ``play pretend'', meaning that they have a hard time imitating the actions of others when playing and therefore prefer to play alone. Furthermore they have difficulties with social interaction and communication -- verbally and non-verbally.

People diagnosed with autism may;

%% Every person with autism is different, just like normal people, however in general people with autism may:

\begin{itemize}
\item Be very sensitive to light, noise, touch, and taste.
\item Have a hard time adjusting to new and changing routines.
\item Show unusual attachments to objects.
\end{itemize}

Autism diagnosed individuals may have a hard time starting and maintaining a conversation. They may communicate with gestures instead of words, develop language slower or faster than normal and some do not develop any language at all.
Furthermore the lack of social interaction means they might have a hard time making friends, may be withdrawn and may avoid eye contact.\citep{autism}

\subsection*{Signs and tests}
\label{sub:signsAndTests}
If a child fails to meet any of the following language milestones, it may be an indication that it needs to be tested for autism;

\begin{itemize}
\item Babbling by 12 months.
\item Gesturing (such as pointing or waving goodbye) by 12 months.
\item Saying single words by 16 months.
\end{itemize}

Children failing to meet any of the previous mentioned language milestones might receive a hearing evaluation, a blood test and a screening test for autism. Since autism covers a broad spectrum of symptoms, a single brief evaluation cannot predict what abilities the child has. Therefore a range of different skills are evaluated, such as:

\begin{itemize}
\item Communication
\item Language
\item Motor skills
\item Speech
\item Success at school
\item Thinking abilities
\end{itemize}

Some parents might be scared of having their child diagnosed, %because they are scared of labelling what is wrong with their child,
however without a diagnosis, the child might not get the necessary help.\citep{autism}

\subsection*{Treatment}
\label{sub:treatment}
Autism cannot be cured, however an early diagnosis and treatment can greatly improve the child's quality of life. Different treatment programs usually build on the child's interests and are highly structured to their needs and routines.\citep{autism}
