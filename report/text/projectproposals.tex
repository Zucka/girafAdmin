\chapter{Project Proposals}
\label{chap:projectProposals}
To keep a conteius progress of GIRAF is there need for futher development oppeunities there have written down projects we have fund though our time working with the GIRAF project.   

\textbf{Event Calender} 
After meeting with Mette Als at Birken was it clear that there is need for digital adminstration orgenice tool the day basis orginas the guardiens and children main activities.
The Schema should also be convertet to a simplifieded version for parants.   

\textbf{Daily Sekvens}
There is need for a visiule daily sekvens for children that is interaktive.     

\textbf{Child Mode}
Child mode is needed to ensure childeren is able to play with the tablet without supervion of gauidians.
Child mode means that the tablet should be total rescrictet to a special app or special apps in launch mode.     

\textbf{Data and Analysis}
Create application that gathers information from apps to analyse the childs development. 

\textbf{Scheduled Sekvens}
Expand sekvens to be triggered by events like time on day or a activity.

\textbf{Co-op Games}
Develop a game that that has in focus for children to inter act with other children to accomplish a task.

\textbf{Development Game Tool}
Create a tool for easy game development for children with autism.  

\textbf{Language Stimulatioen}
Create a apps that stimulate children's ability to spell and enunciate words.       

