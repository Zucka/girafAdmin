\section{Project Proposals}
\label{chap:projectProposals}
To keep a continues progress of GIRAF there is need for further development opportunities. Here there is have written down projects we have found through our time working with the GIRAF project.   

\textbf{Event Calender} 
After meeting Mette Als at Birken it was clear that there is need for a digital organizational tool. To plan the guardian and the children's daily schedule.\\
The Schema should also be able to print for parents.     

\textbf{Child Mode}
Child mode is needed to ensure children is able to play with the tablet without supervision from guardians.
Child mode means that the tablet should be totally restricted to a special app or apps in launch mode.     

\textbf{Data and Analysis}
Create application that gathers information from apps to analyze the child’s development. 

\textbf{Co-op Games}
Develop a game that has in focus for children to interact with other children to accomplish a task.

\textbf{Development Game Tool}
Create a tool for easy game development, so that guardians, is able to design games for the children.

\textbf{Language Stimulation}
Create an app that stimulate children's ability to spell and pronounce words.       

