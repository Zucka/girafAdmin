\chapter{Conclusion}
\label{chap:conclusion}
This projects focus was on making an administrative interface for the GIRAF system on both a desktop and an android platform. Although the developed system was never deployed on an android platform the system is running on a server and is accessible through common web-browsers.\\
As it was already shown in chapter \vref{chap:systemOverview} some of the features in the system were not completed. This was due both to trouble with the Wasteland API as well as unforeseen trouble with different aspects of the already created system.\\
This project did however not set out to make a 100\% complete system, but rather a system that, included completed parts, which would not require additional work. With that in mind the most important parts to complete of the administration system would be the user management parts, since the pictogram features would be implemented by other projects on the tablet.\\
\begin{table}[!h]
        \centering
                \begin{tabularx}{\linewidth}{| X | c |}
                        \hline
                        Description & Severity\\\hline\hline
                        It is not possible to create new categories of Pictos & Critical \\\hline
                        Profile - Restrict QR editing to department manager & Critical \\\hline 
                        DB Problem - Not possible for all pedagogues to fix Pictos for all department children & Critical \\\hline
                        Navigation - The site ``Profiles'' should link to each profile & Critical \\\hline
                        Missing - Unable to remove relations & Critical \\\hline
                        Profile Pic - Accept button is hard to find & Serious \\\hline
                        Profile Pic - Word ``change'' is misleading & Serious \\\hline
                        Navigation - ``Add'' and ``Make'' under Pics Manager is confusing & Serious \\\hline
                        Profile - Department should be a link, not editable & Serious \\\hline
                        Profile - Links from Own Profile relations is missing & Serious \\\hline
                        Profile Pic - Word ``Edit'' is not informative & Cosmetic \\\hline
                        DB Problem - Own Profile takes too long to load & Cosmetic   \\\hline
                        Standardise button names & Cosmetic \\\hline
                        Logout - Can navigate in system without session & Cosmetic \\\hline
        \end{tabularx}
        \caption{Bugs/Errors Found Under Usability Testing without the fixed bugs}
        \label{tab:Bugs/ErrorsMinusTheDone}
\end{table}

By looking at table \ref{tab:Bugs/ErrorsMinusTheDone}, which displays all the errors and bugs we found under the usability test described in chapter \vref{chap:usabilityTesting}. It is clear that there are a number of critical errors in the system. Which in term makes the system unable to fulfil its goal. In order to fulfil the goal these errors must be corrected.
4 out of 5, of the critical errors can be corrected by a couple of days hard work, but the last critical error regarding the database problem, can not be corrected within the administration system.\\
\\
Concluding on the usability test the system is quite user friendly. If the problems in table \ref{tab:Bugs/ErrorsMinusTheDone} were corrected the system should be free of slowing effects in the matter of daily use. Multiple of the test persons did praise the system on the simple structure during the interview after the test.\\
\\
The system only have \underline{one} codebase and the maintainability for the system should therefore be a lot higher than that of the former two systems. Also the group have made an effort to both comment on the written code, and in this report describe some of the more complex solutions, especially in chapter \vref{chap:systemOverview}. This report also comes with an install guide in appendix \vref{chap:installGuide}, so that this project should be as pain free to continue on, as possible.\\
\\
With all of this the initial goal is not obtained. But much has been done to ensure that continuing on this project would be easy. A goal we thought higher then simply completing the system.\\
\\
In the next sections we will evaluate on the multi-project itself and after that make a note of what could be done in the future to improve on this system as well as the GIRAF system as a whole.
