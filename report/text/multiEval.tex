\section{Evaluation of the Multi group project}
\label{sec:multiEval}
This semester is a special semester for software students at Aalborg University. This semester is the first time we work as teams on a bigger project. In connection with this we wanted to evaluate the multi project as a whole, so that future students might benefit from our findings.\\
\\
\subsection{Cooperation between groups}
% Folk undgik at deltage, hvis de kunne
First of all, it is always hard to get along when you are 8 groups consisting of nearly 30 people, that each have their own agenda in mind. That was how we experienced the multi project. We did all agree on one common goal, but it seemed more like people preferred to isolate themselves to their own projects. - This might have to due with the fact, that we up until now have worked that way.\\
\\
%Samarbejde = Nogenlunde - Gik galt med DB gruppen midt i projektet
But the few groups that was entirely dependent on each other, for example our group and the database group, at least tried to work together. However the database group was faced with much hardship, both from their project and some -of the other groups within the multi project, and as a result of this ended up being hard to contact for a certain, important, period of time.\\
\\
%Første projekt der mindede om hvordan det vil være i virkeligheden efter UNI
However this was an interesting experience, which have taught us much about how group work functions outside the university. How big groups of people have to work together in order to complete a vast system. Both with its ups and downs.

\subsection{Advice for next year}
As we have been working with the multi projekt for about 6 months, we have some advice and suggestions for the coming year.\\
\\
\textbf{Advice}\\
% Advice: Bestem fokus for projekt tidligt
In order to end up with a more satisfiable product, decide on the focus of the project, early. We decided the focus for our project, in the middle of it, which turned out to be too late. We feel that we made an okay product, but if we had decided on a focus earlier. It might have been better.\\
\\
% Advice: CI,Redmine,Wiki,GitHub = God base, bør beholdes
This multi project, was so lucky to have someone that was able to set up a starting suit of software. We started out with having a server with CI, Redmine and a Wikipedia. As well as the decision to force all projects on GitHub, have made it very easy to find and work with each others project. We highly recommend keeping this going for the next years students.\\
\\
\textbf{Suggestions}\\
There was a few things that we would have liked to try this semester, but we did not, for whatever reason. Which we here list.
%Opfordring: Mere showcase!
%Opfordring: Mere interaction mellem grupperne, måske ikke fest? Men få folk ind i samme grupperum
%Opfordring: Fælles usability test

\begin{itemize}
	\item Make it a habit to show off your system, so that the other groups might get inspiration from it.
	\item Work together more between the groups. Not necessary on the same thing, but in the same room.
	\item When nearing the end of the project. Try to strive for the ability to make a joint usability test, between the groups.
\end{itemize}


\subsection{General Evaluations}
There also was a few things, that did not fit any category. These are explained here.\\
\\
%Burndowncharts - For meget styring
In this multi project, we used burndowncharts, which we ended up thinking of more as a chore then a beneficial work progress. We believe that this is due to the fact that we are not educated properly in their practical use. So if the students next year is going to use burndowncharts, they might want to consider having a joint session about, how to use burndowncharts.\\
\\
% Rapporten virker underlig i dette semester - Måske lave det om til dokumentation, eller tidligere bestemme hvad meningen med rapporten er?
We found it weird, the way we had to construct our report. It felt like a documentation of the system, but in the form of a project report. We are not entirely sure how to fix this problem, but we would suggest that the semester coordinator explained how to think of this report to the students.\\
He did try to do so, this semester by giving a list of things that needed to be in the report, however this did not help on the mindset of the report.\\
\\
% Tablet problem
In our group, we were so unfortunate that our tablet broke down. We had to wait for more than a month of the development time, before receiving it back from repair. It might have been lucky, that it was our group that had this problem. Since we were able to work on our PC version of the system. - However it will become problematic if there is no better backup solution, if a tablet fails in the future.\\
\\
% Multimøde = bedre oversigt over folks progress
% Komite = Nemmere og mere velovervejede beslutninger
This semester, we the multi group, decided to have a meeting each week. Sometimes it might have felt like a waste of time, however it also gave a better overview of the other groups progress. Also it made it harder for problems to go unnoticed.\\
We also sat down committees, that could take some major solutions over the span of the individual groups. We appreciated this much, because it felt like an effective way to make decisions that had great impact.


