\chapter{Transcript of Meeting with the head of the kindergarten (1/3-2013)}
\label{interviewMette}

\textbf{The questions we brought with us:}\\
A: Does the parents have a common contact list between each other, like the ones used in the, Danish, elementary school?\\
B: How often do you change the kids pictograms? - As in all the way out of the scrap book\\
C: How many kids does a pedagogue take care of, on average?\\
D: How is the parents allowed to influence the daily life of their kids, when the kids are at the kinder garden?\\
E: Does the department manager have a list of the e-mails of each parent and are they easy to find? - If not, how easy would it be to gather them?\\
\\
\textbf{Answers to these questions:}\\
A: The parents are not allowed to know anything about each other or the other parents kids.\\
\\
B: It doesn’t happen that often, but they still need to be able to change the pictures really fast. Since sometimes the kids come in with a new jacket, and cannot put their jacket back on, or take it off because of the wrong visual stimuli from an old picture.\\
The same goes if the milk carton suddenly changes to a Christmas theme or such.\\
\\
C: A pedagogue usually have the responsibility for 2-3 kids. - These are called ''kontakt børn''.\\
\\
D: The parents have no influence. Actually if their kids cannot participate in outdoor activities because of illness or tiredness, the parents are told to keep their kids at home if the plan for the day involves outdoor activity, (which they at Bækken do almost every day).\\
\\
E: She has their e-mail, but they are not public accessible. This still makes it possible for her to send invites to the system. - We were informed that the registration forms for when a kid enters the kinder garden has been updated so that the parents are asked to fill in their e-mail address.\\
\\
\textbf{Information gathered for PictoCreater group:}\\
- The text should not be stationary.\\
- There should be an option for drawing lines and removing parts of a pictogram.\\
These two lines were derived from what she told us of their current pictogram editor system.\\
\\
\textbf{Possible future work:}\\
1. A planning tool. The department manager uses a certain amount of time on scheduling the kids and pedagogues day. One of the most serious issues could be that one of her pedagogue had to take the day off, then she must make sure that all her schedules get taken care of. A feature that they could really use would be that of fixing an assignment for each second week.\\
This planning tool should also be able to print two different versions of the schedule. One for the parents and one for the pedagogues. - The parents are not supposed to know which kids do what, or with whom. They only need to know what general activities is taking place in the kindergarten.\\
She also suggested that the taxa schedule could be incorporated in this system.\\
\\
\textbf{Other observations:}\\
''Profiler'' - We noticed that the tab ''Profiler'' from our original design was of no use to anyone but the department manager. Since everyone else only had access to a certain list of persons, that we already displayed on ''Egen Profil''.\\
\\
''Profiler'' - Should be ordered in a table, where first the pedagogues name is, the next <td> should then contain the kids they are responsible for, the next <td> should then contain the parents of this kid, and the last <td> should then contain other relevant persons or information about the kid. (<td> is the same as a cell in a table)\\
\\
''Parent/Pedagogue contact'' - They don't communicate all that much in person. Since a department can have kids from basically all over the country. The kindergarten do host some 'parent, come and see the kindergarten nights'. But outside that they only communicate via phone and this black book the kids bring to and from the kindergarten. In it important information like, if the kid did something special today, or the reason why he arrived home with a new pair of pants on, and so forth. - The pedagogues always check this book when the kid arrives and so does the parents.\\
\\
''Privacy setting'' - The privacy setting private is known as ''mine tavler'' by the pedagogues.\\
\\
''Privacy setting'' - We need 5 different privacy settings: \\
Pedagogues only\\
Parents only\\
Guardians (''værger'') - We need to confirm this phrasing with the head of the kindergarten\\
Department\\
Public\\
\\
''Searching'' - When searching through tags or categories, it should search on synonyms as well.\\
\\
''Breadcrumbs'' - There should be breadcrumbs on every screen, or if there is only one action on the screen, it should contain its title.\\
\\
''Pictogram 'Edit' and 'Opret''' - We forgot to add category and tag adding to these functions.\\
\\
''App Manager'' - She found it intuitive to perform special settings on the apps in the app manager. - This could be an interesting design idea, but hard to implement.











