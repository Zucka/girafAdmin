\section{QR Code Generation and Printing}
\label{sec:qr}
The multi-group last year decided that QR codes would be used as the only login method on tablets, which meant that the GIRAF administration system would need a system for generating and printing new QR codes, in the event a person lost their QR code. In collaboration with Wasteland it was decided that it would be a security risk to allow people to print out existing QR codes, which meant that our system only has to facilitate the generation of new QR codes and the printing of these new QR codes. The last multi-group also decided that the QR codes would be based on a 512 character long string. The system generates the new QR codes using the code displayed in \autoref{lst:qrcode}. The QR code generation uses the function \texttt{microtime} in order to get data to hash. The function \texttt{microtime} relies on the system call \texttt{gettimeofday}, which means that this implementation will only work on UNIX and Windows based systems (PHP supplies its own implementation in Windows). The time from \texttt{microtime} is then hashed with the \texttt{sha512} hashing function which generates a 128 hexadecimal long string. This is then repeated 3 more times in order to get a 512 hexadecimal long string, which is then the newly generated QR code.

\begin{lstlisting}[firstline=1,caption={QR Code Generation},label=lst:qrcode]
function generateNewQr()
{
	$qr = "";
	for ($i=0; $i < 4; $i++) { 
		$time = microtime();
		$qr .= hash("sha512",$time);
		usleep(100); // sleep for 100 microseconds (0.1 milliseconds) to get a different time from microtime
	}
	return $qr;
}
\end{lstlisting}

The newly generated QR code is then inserted into the GIRAF database using the Wasteland database API. Now the user is prompted to print out the new QR code, as this is the only chance the user has, without generating yet another new QR code. The PHP library \texttt{phpqrcode}\citep{phpqrcode} is used to generate the QR code itself into an image. In order to support the scalability of the QR code, the image is generated as a SVG, but because the original implementation of \texttt{phpqrcode} does not support SVG output, a modification of \texttt{phpqrcode} is used\citep{phpqrcodet0k4rt} which adds support for SVG and EPS. Due to a bug with Internet Explorer, \texttt{phpqrcode} was further modified so that the colour of the QR code is statically black. The bug with Internet Explorer was that \texttt{phpqrcode} would truncate the hex code of the colour black to \#0 which in Internet Explorer would display as white, while in other browsers it would display as black. \\
After \texttt{phpqrcode} has generated the SVG, it is then added to a hidden iframe, which only contains the SVG and a separate CSS file which contains the style for printing the QR code. JavaScript is then used to open the printing dialogue of the browser with the iframe as its focus, so that only the QR code is printed and not the whole page.
