\chapter{Project Progress}
\label{sec:projectProgress}
%Why do we explain this?
This chapter concerns the work progress of the project and the documentation here of. The work progress was split into weekly progress which was documented with a weekly abstract which is to be read in conjunction with sprint story. At the end of this chapter the reader should have the knowledge of the development structure which is used and have the overview of the work that was done in this project. 


\section{Weekly Abstracts}      
\label{subsec:weeklyAbstracts}
%Why did we make Weekly Abstracts?
Weekly abstracts are a small resume on what have been developed over a week and what problems occurred during that time. Usually the weekly abstract is between 5-10 lines of text which is written at the end of each week. Writing weekly abstracts is an easy way to ensure that the development move forward because if there is nothing to write, nothing was made and one should reflect upon that. Also newcomers or an involved third person would be able to quickly get updated on project status by reading the abstracts.         

\section{Weekly Progress}
\label{subsec:weeklyProgress}
%Why did we do what we did?
In the rest of the section is a description of the weekly progress in conjunction with sprint story.
The conclusion of the work progress will be at the end of this chapter. 

The first two weeks of the project were spent to create the groups and determine which projects should be worked on this year.  
Internal in the group the expectations were aligned and work conditions were established.
Graphic figures, including logo, and design guidelines were also established.      

Sprint stories were first used in week 3 (04/03/2013-08/03/2013).   

\textbf{Week 1 (18/02/2013-22/02/2013)} 
Designs and features for the Admin interface were discussed.
Low-fi board designs were made to give abstract view of our ideas regarding the meeting with our contact person.   
The first considerations to incorporation of the multi language support.
The structure of the PHP code, the project and the report had been made so it was ready for the first sprint.
Setup GIT and Redmine.
It was agreed upon to make a web based administration system which is convertible to both desktop as well and  tablet.  
\textit{For further information this choice is readable in chapter \vref{chap:systemDesign}.}

\textbf{Week 2 (25/02/2013-01/03/2013)}
The focus has been on finishing the preliminary interface design and web application structure.
All of the interface designs which were drawn on whiteboards were converted into Balsamiq mock-ups for a coherent look and feel.
At the meeting with the contact person there was feedback on the design and the features. 
Work in committees started. 
\textit{The feedback, questions and answers regarding how the day care works can be found in appendix \vref{interviewMette}.
The Balsamiq mock-ups and web application structure is found in appendix \vref{apx:mockups}.}

\textbf{Week 3 (04/03/2013-08/03/2013)}
This weeks work has been on implementing the design and the functionality of the login screen and main navigation.  
We have also edited our mockup designs with the feedback we got from our contact person. 
Bootstrap is included to streamline the design. 

The first SCRUM sprint  started with the main story being:

\textit{"The guardian is in the launcher and starts all the applications one after another, the guardian can freely move from application to launcher at any given time.
The guardian also enters the administration."}

\textit{which lasted from 04/03/2013-18/03/2013.} 

\textit{From that story we concluded the demand for our sprint was for a user to log in to the system and view his/her profile page. }

\textbf{Week 4 (11/03/2013-15/03/2013) }
The first scrum sprint is completed. 
Following sites are somehow functional implemented in the Admin system:
\begin{itemize}
        \item Log In/Out
        \item Own Profile
        \item View own information
        \item Edit own information
\end{itemize}
The site can still not be shown outside the campus network, and therefore can it not be shown to the contact person.
Regarding functionality the site is not connected to the final database yet but all the test data are there.  
The first parts of the common report are created.

\textbf{Week 5 (18/03/2013-22/03/2013)}
Most progress was in implementing the design and edit the mock ups. The committee for the DB-API is started. This committee is important as it will define connection to the database. 

The new sprint is:
``The guardian is in a application and is working with a picture''

The focus is therefore on implementing profile pictures. 

\textbf{Week 6 (25/03/2013-29/03/2013)}
Time was short because of Easter. 
Following features were integrated the Profile Picture Edit and ability to Change QR.
Mock-ups of Profiles, Create Profile and Department Management were also worked on.    

\textbf{Week 7 (02/04/2013-05/04/2013)}
This week there were only were 4-8 hours of free work time, because of the many lectures and the holiday of this week.

The new sprint is as follows: 
``The guardian creates a pictogram, and imports the pictogram to an application. The guardian personalizes the application.''

\textbf{Week 8 (08/04/2013-12/04/2013)}
This week an auto update script that fetches the git-repository for our web-server was made.
Uploading the profile picture was reworked into a faster and more data efficient way. 
But still missing the database to store the data and to give a report when the image fails to upload or succeeds.
The QR generators is now fully functional. 
There is now a printer function to send our QR-images to the browser or the OS printing service.
\\
\textbf{Week 9 (15/04/2013-19/04/2013)}
The QR system is changed, so that it is secure. 
Language support now works for all the developed sites. \\

\textbf{Week 10 (22/04/2013-26/04/2013)}
Uploading of profile picture is finished.
The work is now on getting the Admin interface down to the tablet.
We have scheduled a meeting with Mette Als (our contact person), to test our system. 

New sprint: 
``The guardian can navigate between applications and use them''
\\
\textbf{Week 11 (29/04/2013-03/05/2013)}
This week was spent on solving the whole problem with the missing database.
A front end was created for the Picto Admin. Features  in our system cannot be supported in IE9 and below. Therefore a warning were added at the log-in page that informs the user that they are using an old browser if the browser does not support the ``FileReader'' system in JavaScript.

\textbf{Week 12 (06/05/2013-10/05/2013)}
The week was spent to preparing for the usability test(Monday 13th).
Profile Create and Make Relations were finished, with exception of the DB create implementation. 
Also Picto Admin Create is fully finished. But it does not support categories. 
The weekend between the 10th and the 13th was used to ensure that the system could work with the newly developed database API. The system was not ready in time for the usability tests. 

\textbf{Week 13 (13/05/2013-17/05/2013)}
The tests were completed and analysed. 
All the bugs and the wanted changes were documented and set to be fixed.
Most of the week was spent on implementing the DB API in the system and making fixes.     
The report structure was made and writing of the report has been started. 
\textit{The finished functionality can be found in chapter \vref{chap:systemOverview}.}
 
At the end of the project the program was tuned to be fully functional and include notes.
Until the report delivery was writing documentation.       

\section{Conclusion of Project Progress}
The weekly abstracts were a good point to acknowledge the progress and evaluate the work effort. It was also used to inform our supervisor of the progress we had made during the weeks. 
The group work process was more a natural development process forced by the sprint story.       
The one thing that did complicated the work was the lack of cooperation with Wasteland which should have been much closer. 
\label{sec:conclusionofProjectProgress}
