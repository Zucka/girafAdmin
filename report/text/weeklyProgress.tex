\chapter{Project Progress}
\label{sec:projectProgress}
%Why do we explain this?
This chapter concerns the work progress of the project and the documentation here of. 
The work progress was split into weekly progress that was documented with a weekly abstract that is to be conjunction with sprint story.  
At the end of this chapter should the reader have the knowledge of the development structure that is used and have the overview of the work that is done in this project. 


\section{Weekly Abstracts}      
\label{subsec:weeklyAbstracts}
%Why did we make Weekly Abstracts?
Weekly abstracts is a small resume on that has been developed over the week and problems that may have occurred. 
Usually is the weekly abstract between 5-10 lines that is written the end of each week. 
Writing weekly abstracts is a easy way to ensure that the development move forward because there will nothing to write if none is done.
New comers or a involved third person will quickly manage to get up to date with reading the abstract. 
At the end of the project it is possible to extract the problems to improve the development method, use as documentation for project and to overall analysis.          

\section{Weekly Progress}
\label{subsec:weeklyProgress}
%Why did we do what we did?
In the rest of the section is a description of the weekly progress in conjunction with sprint story.
The conclusion of the work progress will be at the end of this chapter. 

The first two weeks of project was used to create the groups and determent projects.  
Internal in the group was the expectation aligned and work conditions was established.
Grafical figures and design guidelines was also established.      

Sprint stories was first used in week 3 (04/03/2013-08/03/2013).   

\textbf{Week 1 (18/02/2013-22/02/2013)} 
Designs and features for the Admin interface was discuessed.
Low-fi board designs was made to give abstract view of our ideas regarding the meeting with our contact person.   
The first considerations to incorporation of multi language support.
The stucture of php, project and report has been made so it is ready for the first sprint.
Setup GIT and Redmine.
It was agreed upon to make a web based administration system that is convertible to both desktop and tablet.  
For further information this choice is readable in chapter \fix{referance} 

\textbf{Week 2 (25/02/2013-01/03/2013)}
The focus has been on finishing the preliminary interface design and web application structure.
All of the interface design that was drawn on whiteboards is converted to Balsamiq mock-ups for a coherent feel.
At the meeting with the contact person there was feedback on the design and features. 
The work in the committees is started. 
The feedback, questions and answers regarding how the day care works can be found in \fix{appendix}.
The Balsamiq mock-ups and web application structure is found in chapter \fix{henvisning} 

\textbf{Week 3 (04/03/2013-08/03/2013)}
This week work has been on implementing the design and functionality of the login screen, main navigation.  
We have also edited our mockup designs with the feedback we got from our contact person. 
Bootstap is included to streamline the design. 

The first scrum sprint started with the main story being  our, which will last from 04/03/2013-18/03/2013. 

"The guardian is in the launcher and starts all the applications one after one, the gaurdian can freely move from application to launcher at any given time.
The guardian also enters the administration."

From that story we concluded the demand for our sprint was for a user to log in to the system and view his/her profile page. 

\textbf{Week 4 (11/03/2013-15/03/2013) }
The first scrum sprint is completed. 
Following sites is somehow functional implemented in the Admin system:
\begin{itemize}
	\item Log In/Out
	\item Own Profile
	\item View own information
	\item Edit own information
\end{itemize}
The site can still not be shown outside of the campus network, and therefore can it not be shown to the contact person.
Regarding functionality is the site not connected to the final database there is all the data dummy data.  
The first parts of the common report is created.

\textbf{Week 5 (18/03/2013-22/03/2013)}
Most progress was in implementing the design and edit the mock ups. The committee for the DB-api is started. This committee is important as it will define connection to the database. 

The new sprint is:
"The guardian is in a application and is working with a picture"

The focus is therefor on implementing profile pictures. 

\textbf{Week 6 (25/03/2013-29/03/2013)}
Time was short because of Easter. 
Following features was integrated the Profile Picture Edit and ability to Change QR.
Mock ups of Profiles, Create Profile and Department Management was also worked on.    

\textbf{Week 7 (02/04/2013-05/04/2013)}
This week there only were 4-8 hours of free work time, because of the many lectures and the holiday of this week.

The new sprint is as follows: 
"The guardian creates a pictogram, and imports the pictogram to an application. The guardian personalizes the application."

\textbf{Week 8 (08/04/2013-12/04/2013)}
This week an auto update script that fetches the git-repository for our web-server were made.
Uploading the profile picture was reworked to a faster and more data efficient way. 
But still missing the database to store the data and to give a report when the image fails to upload or succeeds.
The QR generators is now fully functional. 
There is now a printer function to send our QR-images to the browser or OS printing service.

\textbf{Week 9 (15/04/2013-19/04/2013)}
The QR system is changed, so that it is secure. 
Language support now work for all the developed sites.

\textbf{Week 10 (22/04/2013-26/04/2013)}
Uploading of profile picture is finished.
The work is now on getting the Admin interface down to the tablet.
A date with Mette Als (the contact person), to test our system.

New sprint: 
"The guardian can navigate between applications and use them"

\textbf{Week 11 (29/04/2013-03/05/2013)}
This week were used on solving the whole problem with the missing database.
A front end was created for the Picto Admin. Features  in our system can not be supported in IE9 and below. Therefor is a warning added at the log-in page that informs the user that they are using an old browser if they do not support the "FileReader" system in JavaScript.

\textbf{Week 12 (06/05/2013-10/05/2013)}
The week was used to prepare for the usability tests(monday the 13th).
Profile Create and Make Relations got finished, with exception of the DB create implementation. 
Also Picto Admin Create is fully finished. But it does not support categories. 
In the weekend between the 10th and the 13th was used to ensure that the syestem could work with the newly developed database API. The system was not ready in time for the usability tests. 

\textbf{Week 13 (13/05/2013-17/05/2013)}
The tests was completed and analyzed. 
All the bugs and wanted changes got documented and set to be fixed.
Most of week was used on implementing the DB API in the system and make fixes.     
The rapport structure was made and writing is started. 
The finished functionality can be found in chapter \vref{chap:systemOverview}.
 
At the end of the project the program was tuned to be fully functional and include notes.
Until the report delivery was writing documentation.       

\section{Conclusion of Project Progress}
The weekly abstracts was good point to acknowledge the progress and evaluate the work effort.
Other the that it not really used. 
The group work process was more a natural development process the forced by the sprint story.       
The one thing that did complicated the work the lack of cooperation that should have been much closer. 
\label{sec:conclusionofProjectProgress}


