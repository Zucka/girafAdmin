\chapter{Installation Guide and Configuration}
\label{chap:installGuide}
This section will describe how to install and configure the GIRAF Admin system. 
\section{Requirements}
\begin{itemize}
\item A suitable operating system (Ubuntu, Windows and OSX was used during development)
\item Apache (Not tested on nginx)
\item PHP5 with the socket module enabled
\item A suitable browser (Chrome,Firefox,Safari,IE10+ is recommended)
\item GIT (The project is hosted on GIT)
\item Python if auto update is used
\end{itemize}
\section{Installation}
\begin{itemize}
\item Install Apache, PHP5, GIT and Python
\item Pull the most current version from https://github.com/Zucka/girafAdmin (although forking the project is advised)
\item Copy the contents of source/desktop into the www root of Apache (usually /var/www/ on UNIX systems)
\end{itemize}
\section{Configuration}
\begin{itemize}
\item Enable the socket module of PHP5 if not already enabled
\item Enable the header module of Apache if not already enabled
\item Modify the Apache configuration file (httpd.conf) to auto-inject a meta tag into every header (used to force Internet Explorer into standards mode). The configuration fragment to be added is listed in \autoref{lst:apacheheader}. 
\begin{figure}[htbp]
\begin{lstlisting}
<IfModule headers_module>
   Header set X-UA-Compatible: IE=Edge
</IfModule>
\end{lstlisting}
\caption{The configuration fragment to add to httpd.conf}
\label{lst:apacheheader}
\end{figure}
\item Change the variables address and port in source/desktop/db/new.db.php to reflect the IP address and port of the server that hosts the database API
\end{itemize}
\section{Auto Update}
A python script was made to auto update the server from a GIT. It can be used to auto pull new changes from a GIT and then move the contents of a folder ínto another folder. To get it to work you need either Python 2.X or 3.
\begin{itemize}
\item Copy autoupdate.py from source/scripts/autoupdate.py to a seperate folder, to be used as an intermediate folder
\item Change all references to \texttt{/home/neo/autoupdate} to point to the seperate folder created for auto update
\item Change all references to \texttt{https://github.com/Zucka/girafAdmin.git} to point to the new GIT for the project (the GIT installed on the system needs to have read rights for the GIT repository)
\item Give autoupdate.py execute rights (sudo chmod +x autoupdate.py)
\item Make sure that the www folder and all subfolders have the same owner
\item Start autoupdate.py as the user that owns the www folder (example: \texttt{sudo -u www-data python autoupdate.py})
\item Alternatively start autoupdate.py in a screen (the UNIX program) instance so that auto update keeps running after the ssh connection has been terminated
\item The www folder will now be updated every 60 seconds with the newest changes from the specified GIT
\end{itemize}