\chapter*{Introduction}
\label{chap:introduction}
\vspace{-2.0em}
This project is part of the multi-project called GIRAF, which stands for: ''Graphical Interface Resources for Autistic Folk'' which have been an active project at Aalborg University since 2011.\\
As the name suggests this system is a tool made for helping kids with autism to communicate. The system is to be actively used in the specialized kinder gardens and schools here in Aalborg.\\
This report contains first an introduction to the GIRAF project as a whole, which describes the GIRAF project as it is of the day this report was written. This introduction is contained in all project reports for the GIRAF multi-project, and can therefore be skipped if the reader already is familiar with the GIRAF project as of year May 2013. The place to be skipped to is then part \vref{skipToMe}.\\
Second this report also contains the thoughts that went into constructing the new administration system, as well as some in depth going explanation of the implementations, for the GIRAF system.\\
The administration system makes it possible to edit and create users for the GIRAF system and in its finished form is possible to be controlled both from a tablet and from a PC over the internet. It also comes with the possibility to make, edit and manage pictograms on a PC over the internet.\\
\\
In this report we also use some terms that often is misunderstood, as well as a few special words, we therefore want to clarify them here before we begin. The wording is taken from wikipedia \citep{wikipedia}.


\begin{table}[h]
	\centering
		\begin{tabularx}{\textwidth}{|l|X|}
			\hline
			Word & Meaning\\\hline\hline
			Website & A set of related webpages served from a single web domain\\\hline
			Webpage & A single page displayed on a website\\\hline
			Homepage & The initial or main web page of a website\\\hline
			Site & Refers to the word website\\\hline
			Pictogram & A special element, developed for the GIRAF project. Containing sound, text and image\\\hline
			Guardian & A person with responsibility for a child with autism, either a pedagog or a parent\\\hline
		\end{tabularx}
	\caption{Word Explanation}
	\label{tab:WordExplanation}
\end{table}
